\documentclass[11pt,a4paper]{scrartcl}

\usepackage[utf8]{inputenc}
\usepackage[T1]{fontenc}
\usepackage[ngerman]{babel}
\usepackage{amsmath,amsthm,amssymb,dsfont}
\usepackage{mathtools}
\usepackage[paper=a4paper,left=25mm,right=25mm,top=25mm,bottom=25mm]{geometry}
\usepackage{float}
\usepackage{hyperref}
\usepackage{enumerate}
\usepackage{url}
\usepackage{tikz}
\usepackage{esint}
\usepackage{csquotes}
\usepackage{textcomp}

\usepackage{setspace}

\parindent 0pt
\linespread{1.5}

% Makros

\newcommand{\N}{\mathbb{N}} % natuerliche Zahlen
\newcommand{\Z}{\mathbb{Z}} % ganze Zahlen
\newcommand{\Q}{\mathbb{Q}} % rationale Zahlen
\newcommand{\R}{\mathbb{R}} % reelle Zahlen
\newcommand{\K}{\mathbb{K}} % Körper
\newcommand{\C}{\mathbb{C}} % komplexe Zahlen
\newcommand{\D}{\mathcal{D}}
\newcommand{\E}{\mathcal{E}}
\newcommand{\Hc}{\mathcal{H}}
\newcommand{\Sc}{\mathcal{S}}
\newcommand{\Kc}{\mathcal{K}}
\newcommand{\A}{\mathcal{A}}
\newcommand{\B}{\mathcal{B}}
\newcommand{\G}{\mathcal{G}}
\newcommand{\Lc}{\mathcal{L}}
\newcommand{\M}{\mathcal{M}}
\newcommand{\Nc}{\mathcal{N}}
\newcommand{\F}{\mathcal{F}}
\newcommand{\Rc}{\mathcal{R}}
\newcommand{\norm}[1]{\|#1\|}
\newcommand{\laplace}{\triangle}
\newcommand{\circum}{\text{\textasciicircum}}

% Umgebungen für Definitionen, Sätze, usw.

\theoremstyle{plain}
\newtheorem{thm}{Satz}[section]
\newtheorem*{lem}{Lemma}
\newtheorem{cor}[thm]{Korollar}
\newtheorem{prop}[thm]{Proposition}
\newtheorem*{ex}{Beispiel}
\newtheorem*{ntion}{Notation}

\theoremstyle{definition}
\newtheorem{defn}[thm]{Definition}

\theoremstyle{remark}
\newtheorem*{bem}{Bemerkung}
\newtheorem{bemnumber}[thm]{Bemerkung}

\def\Satzrefname{Satz}

\DeclareMathOperator{\supp}{supp}
\DeclareMathOperator{\esssupp}{ess supp}
\DeclareMathOperator{\essrange}{ess range}
\DeclareMathOperator{\id}{id}
\DeclareMathOperator{\loc}{loc}
\DeclareMathOperator{\pv}{pv}
\DeclareMathOperator{\grad}{grad}

\begin{document}

\title{Zusammenfassung Spektraltheorie und Operatoralgebren}
\author{Sebastian Bechtel}
\maketitle

\section{Grundlegendes zu Algebren}

\subsection{Beispiele von Algebren}

% TODO schreiben

\subsection{Elementare Eigenschaften}

Sei $\A$ Banachalgebra. Dann ist die Multiplikation stetig. Hat $\A$ eine Involution und ist die C*-Eigenschaft erfüllt, so ist $\A$ eine Banach-*-Algebra: $\|x\|^2=\|x^*x\|\leq \|x^*\|\|x\|$, also $\|x\|\leq \|x^*\|$, somit $\|x\|\leq \|x^*\|\leq \|(x^*)^*\|=\|x\|$. Ist $1\in\A$, so gilt $\|1\| \geq 1$, denn $\|1\|\leq \|1\|^2$ und ist $\A$ C*-Algebra, so gilt $\|1\|=1$, denn es gilt $1^*=1$ und somit folgt die Behauptung aus $\|1\|=\|1^*1\|=\|1\|^2$.

\subsection{Algebren ohne Eins}

Algebren wie $L^1(\R)$ und $C_0(\R)$ haben keine Eins. Um trotzdem Spektraltheorie betreiben zu können, betten wir sie als Ideale in eine Algebra mit Eins ein.

Algebraisch: $\tilde \A \coloneqq \A \times \C$ mit geeigneter Multiplikation ist Algebra mit $1_{\tilde \A} = (0,1)$ und $\A\ni x\mapsto (x,0)\in \tilde \A$ bettet $\A$ als Ideal in $\tilde \A$ ein.

Banach-algebraisch: Statte $\tilde \A$ mit $l^1$ Norm der direkten Summe aus, also $\|(x,\alpha)\|_{\tilde \A} = \|x\|+|\alpha|$.

C*-algebraisch: Problem: Banach-algebraische Konstruktion erhält C*-Eigenschaft im Allgemeinen nicht. Definiere deshalb andere Norm, deren Konstruktion aber bereits die C*-Eigenschaft benutzt!

\subsubsection{Linksreguläre Darstellung}

Betrachte den Algebrahomomorphismus $\A\ni x\mapsto L_x\in \Lc(\A)$, wobei $L_x(y)\coloneqq xy$ die \emph{Linksreguläre Darstellung} ist. Ist $\A$ C*-Algebra, dann gilt $\|L_x\|=\|x\|$, denn $\|L_x\|\geq \|xx^*\|/\|x^*\|=\|x^*\|=\|x\|$ und $\|L_x\|\leq \|x\|$ ist klar.

Also: Ist $\A$ C*-Algebra, dann betrachte auf $\tilde \A$ die Norm $\|(x,\alpha)\|=\|L_x+\alpha\|_{\mathrm{op}}$. Definitheit nutzt Isometrie des Algebrahomomorphismus und die Tatsache, dass $\A$ keine Eins besitzt.

\subsection{Spektraltheorie in Banachalgebren}

Sei $\A$ Banachalgebra mit Eins. Für $x\in\A$ definiere durch $\rho(x)\ni \lambda \mapsto r(\lambda,x)\coloneqq (\lambda-x)^{-1}$ die \emph{Resolvente} $r(\cdot,x)$ von $x$.

Ist $0\neq \lambda \in \sigma(x)$, so ist $1/\lambda \in \sigma(x^{-1})$. Auch im Fall $xy\neq yx$ stimmen deren Spektren fast überein, es gilt $\sigma(xy)\cup \{0\} = \sigma(yx)\cup \{0\}$. Hinzunahme der $0$ ist notwendig: Sei $S$ der Rechtsshift, dann $S^*S=\id$, also $0\not\in \sigma(S^*S)$, aber $SS^*$ nicht injektiv. Ist $\|x\|< 1$, dann ist $1-x$ invertierbar und die Inverse ist gegeben durch die \emph{Neumann-Reihe} $\sum_{n=0}^\infty x^n$, vgl. geometrische Reihe! Die Resolventenmenge ist offen, also $\sigma(x)$ abgeschlossen, außerdem gilt für den \emph{Spektralradius} $r_\sigma(x)\coloneqq \sup \{ |\lambda|: \lambda \in \sigma(x) \}$ die Abschätzung $r_\sigma(x) \leq \|x\|$, also ist das Spektrum $\sigma(x)$ kompakt. Es gilt $\lim_{|\lambda|\to \infty} r(\lambda,x)=0$, vgl. mit Resolventenabschätzungen für H.G.en, z.B. $\|r(\lambda,A)\|\leq M/|\lambda|$ für Generatoren von analytischen H.G.en. Die Resolvente ist holomorph (leite Potenzreihenentwicklung aus Neumann-Reihe ab). Es folgt $\sigma(x)\neq \emptyset$: Wäre $\sigma(x)$ leer, dann wäre die Resolvente eine ganze Funktion. Wegen dem Grenzverhalten für $|\lambda|\to \infty$ folgt Beschränktheit, also nach Liouville $r(\lambda,x)\equiv 0$, aber $0$ ist nicht invertierbar, Widerspruch. Ist $\A$ Banach-*-Algebra, so gilt $\sigma(x^*)=\overline{\sigma(x)}$, nutze dazu $1^*=1$.

\subsection{Der Satz von Gelfand-Mazur}

Ist $\A$ Banachalgebra mit Eins und jedes $x\neq 0$ sei invertierbar (genannt \emph{Divisionsalgebra}), dann gilt $\A\cong \C$. Bew: Identifiziere $x$ eindeutig mit einem Skalar: Aus $\sigma(x)\neq\emptyset$ folgt die Existenz eines $\lambda$ mit $\lambda-x$ nicht invertierbar, also nach Voraussetzung $x=\lambda$.

\section{Gelfandtheorie}

Ist $E$ normierter Raum, so ist $K\coloneqq (E')_1$ eine schwach-*-kompakte Menge und $E\ni x \mapsto \hat x \in C(K)$ ein isometrischer Isomorphismus von normierten Räumen, wobei $K$ also kompakter Hausdorffraum ist. Für eine kommutative, unitale Banachalgebra $\A$ ist aber $C(K)$ keine Algebra, denn im Allgemeinen gilt $(\hat x \hat y)(\varphi)=\varphi(x)\varphi(y)\neq\varphi(xy)=\widehat{xy}(\varphi)$ (sofern $\varphi$ keine multiplikative Linearform ist, z.B. das Integral auf $L^1([0,1])$).

Ansatz: Schränke $K$ ein, sodass Multiplikativität erfüllt ist!

Wir definieren das \emph{Spektrum der Algebra} via $\hat \A \coloneqq \{ \varphi \in \A^*: \varphi\neq 0, \varphi \text{ multiplikativ} \}$. Dies ist der Kandidat für die Einschränkung. Wir benötigen $\varphi\neq 0$, da sonst $\hat 1$ keine Eins in $C(\hat \A)$ sein kann, denn es gilt $\hat 1(0)=0(1)=0$.

Ist $\varphi\in \hat\A$ und $\A$ unital, so gilt $\varphi(1)=1$, denn $\varphi(1)=\varphi(1)^2$ und $\varphi(1)=0$ impliziert $\varphi=0$. Außerdem gilt $\|\varphi\| \leq 1$, insbesondere $\hat\A \subseteq (E^*)_1$ (fällte also unter Banach-Alaoglu!): Aus $\varphi(x)=\lambda$ folgt $\lambda \in \sigma(x)$, denn dann gilt $\varphi(\lambda-x)=0$ und wegen Multiplikativität und $\varphi(1)=1$ kann dann $\lambda$ nicht in der Resolventenmenge sein. Daraus folgt dann $|\varphi(x)|\leq r_\sigma(x)\leq \|x\|$.

\subsection{topologische Eigenschaften von $\hat\A$}

Ist $\A$ unital, so ist $\hat\A$ kompakt, denn Multiplikativität wird automatisch erhalten und wegen $\varphi(1)=1$ ist der Grenzwert nicht $0$, also ist $\hat\A$ abgeschlossen. Ansonsten gilt $\hat{\tilde\A} \cong \hat \A \cup \{\varphi_\infty\}$, also ist $\hat{\tilde\A}$ die Einpunktkompaktifizierung von $\hat\A$ und somit lokalkompakt.

\subsection{Idealtheorie in $\A$}

Ernte der Idealtheorie wird sein, dass wir $\|\hat x\|=r_\sigma(x)$ erhalten. Dazu werden wir nutzen, dass nicht invertierbare Elemente in echten, maximalen Idealen enthalten sind, die wiederum zu den Kernen von Elementen in $\hat \A$ korrespondieren.

Ist $\A$ Banachalgebra mit Eins, dann ist der Abschluss eines echten (zweiseitigen) Ideals wieder ein echtes Ideal, nutze, dass, falls $I$ dicht, $1\in \A$ impliziert, dass es $y\in I$ gibt mit $\|1-y\| < 1$, also $y$ invertierbar und somit $1\in I$, Widerspruch. Z.B. ist $\F(\Hc) \subset \B(\Hc)$ echtes Ideal und $\overline{\F(\Hc)}=\Kc(\Hc)$ ebenso. Ohne Eins gilt dies nicht, z.B. $C_c(\R) \subset C_0(\R)$ echtes Ideal, aber dicht. Maximale Ideale sind abgeschlossen, denn sonst wäre der Abschluss ein echtes Ideal, dass größer ist, und nach Zorn ist jedes Ideal in einem abgeschlossenen, maximalen Ideal enthalten (ohne Eins mit gleichem Beispiel wie oben falsch). 

Für $\A$ unital, kommutativ ist $x$ nicht invertierbar genau dann, wenn $x$ in einem echten, abgeschlossenen Ideal enthalten ist. Ist $x$ invertierbar, so kann es nicht in einem echten Ideal enthalten sein, da sonst $1=xx^{-1}$ im Ideal wäre. Andererseits ist wegen Kommutativität $x\A=\A x$ zweiseitiges Ideal, $1\not\in\A$, da $x$ nicht invertierbar und wegen $1\in \A$ gilt $x\in x\A$.

Es gilt \emph{der Satz, der die Theorie zum Laufen bringt}: Für ein echtes Ideal $I$ ist Maximalität äquivalent zu $\A/I\cong \C$: Ist $I$ maximal, so enthält $\A/I$ keine echten Ideale, also ist $\A/I$ Divisionsalgebra und nach Gelfand-Mazur isomorph zu $\C$. Die Abbildung $\hat \A \ni \varphi \mapsto \Nc(\varphi) \in M(\A)$ ist Bijektion ($M(\A)$ bezeichne die echten, maximalen Ideala von $\A$). 

\underline{In Summe}: Es gilt für $x\in \A$, dass $\lambda\in\sigma(x)$ genau dann, wenn es $\varphi\in\hat\A$ gibt mit $\varphi(x)=\lambda$, denn: $\lambda\in\sigma(x)$ gdw. $\lambda-x$ nicht invertierbar gdw. es echtes, maximales Ideal $I$ gibt mit $\lambda-x\in I$ gdw. es ex. $\varphi \in \hat \A$ mit $\lambda-x \in \Nc(\varphi)$ gdw. es ex. $\varphi \in \hat \A$ mit $\varphi(x)=\lambda$. 

\subsection{Der Satz von Gelfand}

Ist $\A$ kommutative Banachalgebra, $K\coloneqq \hat \A$, dann ist $\A\ni x \mapsto \hat x \in C_0(K)$ kontraktiver Algebrahomomorphismus und heißt \emph{Gelfand-Transformation}. $K$ ist lokalkompakt und genau dann kompakt, wenn $\A$ unital. Es gilt $\sigma(x)=\hat x(\hat \A)$ und somit $\|\hat x\|_\infty = r_\sigma(x)$.

\subsection{Beispiel: Ideale und Spektrum einer nichtkommutativen Algebra (Ü.A. 13)}

Betrachte die nichtkommutative *-Algebra $\A$ der $n\times n$-Matrizen. Diese besitzt nichttriviale Links- und Rechtsideale (mit Nullzeile bzw. mit Nullspalte), aber keine nichttrivialen zweiseitigen Ideale (sonst forme $M\neq 0$ zu Matrixeinheit um und erhalte $1_\mathcal{A} \in I$). Daher ist $\hat \A = \emptyset$, denn sonst würde der Kern ein nichttriviales zweiseitiges Ideal liefern.

\subsection{Anwendung: Satz von Wiener}

Betrachte die Banachalgebra $(l^1(\Z),*,\|\cdot\|_1)$. Diese hat das Einselement $\delta_0$ und es gilt $\delta_k = \delta_0^k$, somit sind Elemente aus $\hat \A$ durch ihren Wert auf $\delta_0$ eindeutig festgelegt. Für $\alpha \in \mathbb{T}$ definiere $\varphi_\alpha: l^1(\Z) \ni \sum_n f(n) \delta_n \mapsto \sum_n \alpha^n f(n) \in \C$, dann ist $\mathbb{T} \ni \alpha \mapsto \varphi_\alpha \in \hat \A$ ein Homöomorphismus, $\mathbb{T} \cong \hat \A$. Die Gelfand-Transformation ist die diskrete Fouriertransformation (ersetze $\alpha$ durch $e^{i\psi}$ in der Darstellung von $\varphi_\alpha$) und das Bild $\omega$ heißt \emph{Wiener-Algebra} (ausgestattet mit $\|g\|_\mathrm{W} \coloneqq \|\check g\|_{l^1(\Z)}$ anstatt Supremumsnorm).

% TODO Satz von Wiener nicht ganz klar. Was ist f, was ist \hat f? Vgl. Wikipedia!

\section{Spektraltheorie in C*-Algebren}

Ist $x$ s.a., so gilt $\sigma(x)\subset \R$. Ist $x$ unitär, so gilt $\sigma(x)\subset \mathbb T$.

Für $x$ normal gilt $\|x^2\|=\|x\|^2$, via Darstellung des Spektralradius folgt $\|x\|=r_\sigma(x)$ und für allgemeines $x$ gilt $\|x\|=r_\sigma(x^*x)^{1/2}$ (vgl. Spektralnorm von Matrizen).

\underline{Also}: Verknüpfung zwischen Topologie (Norm) und Algebra (Spektralradius)! Somit gibt es auf einer *-Algebra höchstens eine C*-Norm.

Ist $\B$ Banach-*-Algebra, $\A$ C*-Algebra und $\pi: \B \to \A$ ein *-Homomorphismus, so ist $\pi$ stetig mit $\|\pi\| \leq 1$ (algebraische Eigenschaft \enquote{schenkt} topologische Eigenschaft der Stetigkeit!). Beweisidee: Durch Übergang zu $\tilde \B$ und $\A_\mathrm{red}\coloneqq \pi(1)\tilde \A \pi(1)$ (vgl. vNA Reduktion von Algebren) ist $\pi$ oBdA einserhaltender *-Homomorphismus, $x$ invertierbar in $\B$ impliziert $\pi(x)$ invertierbar in $\A$, dann $\|\pi(x)\| \leq \|x\|$ via Abschätzung der Spektralradien.

\subsection{Gelfandtheorie für C*-Algebren}

Ist $\A$ unitale, kommutative C*-Algebra, so ist jedes Element insbesondere normal, also $\|\hat x\| = r_\sigma(x) = \|x\|$, also Bild des Gelfand-Homomorphismus abgeschlossene *-Unteralgebra stetiger Funktionen auf $\hat \A$, die die Eins erhält (wegen $\varphi(1)=1$ für $\varphi\in\hat \A$) und Punkte trennt ($\varphi\neq \psi \in \hat \A$, dann ex $x\in \A$ mit $\varphi(x)\neq \psi(x)$, also $\hat x(\varphi)\neq \hat x(\psi)$), somit nach Stone-Weierstraß $\{ \hat x: x\in \A \} = C(\hat \A)$, also $\A \cong C(\hat \A)$. Hat $\A$ keine Eins, dann $\A \cong C_0(\hat \A)$, $\hat \A$ lokalkompakt.

\subsubsection{Satz von Banach-Stone}

Für kompakte Hausdorffräume $X,Y$ gilt: $X\cong Y$ gdw. $C(X)\cong C(Y)$, nenne daher die Untersuchung von C*-Algebren auch nichtkommutative Topologie. Beweisidee: Zeige $C(\hat \A)=\A=C(X)$ impliziert $\hat \A \cong X$. Setze $\varphi_x(f)=f(x)$, dann zeige $X\ni x\mapsto \varphi_x \in \hat \A$ ist Homöomorphismus. Stetig und injektiv leicht. Surjektivität durch Widerspruchsbeweis: Erhalte nichttriviale Funktion aus Urysohn, die ein Urbild unter dem Gelfand-Homomorphismus in $C(X)$ hat, die trivial ist (Bild!).

\subsection{Spektralsatz und stetiger Funktionalkalkül}

Sei $x\in \A$ normal, $\A$ unital, dann ist die von $x$ und $1$ erzeugte C*-Algebre $C^*(x,1)$ kommutativ, also isomorph zu einer Algebra $C(\Omega)$ mit $\Omega$ kompakt. Es gilt $\Omega \cong \sigma(x)$ (via $\varphi \mapsto \hat x(\varphi) \in \sigma(x)$), somit $C^*(x,1)\cong C(\sigma(x))$ und der Isomorphismus ist durch $i(x)=\id$ eindeutig bestimmt (festgelegt auf Polynomen, diese sind dicht wegen Weierstraß).

Für $f\in C(\sigma(x))$ definiere stetigen Funktionalkalkül via $f(x) \coloneqq i^{-1}(f)$. Da $i$ isometrischer *-Isomorphismus, gilt $(f+g)(x)=f(x)+g(x), (fg)(x)=f(x)g(x), (\bar f)(x)=(f(x))^*, \|f(x)\|=\|f\|$. Es gilt $\sigma(f(x))=f(\sigma(x))$, denn $\sigma(f)=\Rc(f)$.

Anwendung: Ist $x\in \A$ s.a., definiere $e^{ix}\in \A$ unitär. Ist $x \geq 0$, definiere $x^{1/2}$.

\subsection{erzeugte C*-Algebra (Ü.A. 20)}

Ist $T\in \B(\Hc)$ normal, so gilt $C^*(T,1)=\overline{\{p(T,T^*): p\in \C[X_1,X_2]\}}^{\|\cdot\|}$. Jede C*-Algebra, die $1$ und $T$ enthält, muss auch alle Polynome enthalten und abgeschlossen sein und umgekehrt ist die angegebene Menge eine C*-Algebra, die $1$ und $T$ enthält.

\subsection{Spektrum relativ zu einer Algebra (Ü.A. 17)}

Ist $\B \subset \A$ C*-Unteralgebra und $1_\mathcal{A} \in \B$, dann gilt für $x\in \B$, dass $x$ invertierbar in $\A$ genau dann, wenn $x$ invertierbar in $\B$, insbesondere $\sigma_\A(x)=\sigma_\B(x)$.

In (kommutativen) Banachalgebren gilt das nicht: Es ist $l^1(\N_0)$ Unteralgebra von $l^1(\Z)$, die Eins ist in beiden Algebren $\delta_0$, aber $(\delta_1)^{-1}=\delta_{-1}$, also ist $\delta_1$ in $l^1(\Z)$, aber nicht in $l^1(\N_0)$ invertierbar.

\section{Positivität}

Sei $x\in \A$, man nennt $x$ \emph{positiv}, falls $x$ s.a. und $\sigma(x) \subset [0,\infty)$. Jedes Element einer C*-Algebra ist Summe positiver Elemente (zerlege in Real- und Imaginärteil, dort klar im Funktionenbild). Ist $x \geq 0$, so ist die Wurzel $x^{1/2} \geq 0$ eindeutig bestimmt: Ist $y$ mit $y \geq 0, y^2=x$, so ist $x$ und somit $x^{1/2}$ in $C^*(1,y)$, Behauptung folgt mit Eindeutigkeit der Wurzel in $\R$.

Es gilt das \emph{einfach, aber wichtig}-Kriterium: Ist $\A$ unital, $x^*=x\in \A$, $\|x\| \leq 1$, dann sind $x\geq 0$ und $\|1-x\|\leq 1$ äquivalent (Funktionenbild!). Die positiven Elemente $\A_+$ bilden einen abgeschlossenen Kegel (weise das Kriterium nach!).

Für ein s.a. Element $x$ sind $x\geq 0$, $x=y^*y$ für ein $y\in \A$ sowie $x=h^2$ für ein s.a. $h$ äquivalent. Beweisidee: Ist $x \geq 0$, so liefert die Wurzel die gewünschten Darstellungen. Ist $x=y^*y$, so ist $x$ s.a., also zerlege $x$ in Positiv- und Negativteil und zeige $x_-=0$. Viel rechnen, aber es steckt auch viel Spektraltheorie drin, z.B. $\sigma(xy) \cup \{0\} = \sigma(yx) \cup \{0 \}$, Spektralsatz für Zerlegung, $\sigma(x_-^2)=\{0\}$ impliziert $x_-=0$ (Spektraler Abbildungssatz sowie Spektralsatz).

Anwendung: $\A$ C*-Algebra, $x\in \A$ impliziert $x^*x+1$ invertierbar (ehemals Definition für C*-Algebra), denn $x^*x$ positiv, somit $0\not\in \sigma(x^*x+1) \subset [1,\infty)$.

Positivität wird unter Konjugation und *-Homomorphismen erhalten: $x=y^*y\geq 0$, dann $z^*xz=z^*y^*yz=(yz)^*yz \geq0$ und $\pi(x)=\pi(y)^*\pi(y) \geq 0$. Ebenso wird Monotonie erhalten.

\subsection{Produkte positiver Elemente (Ü.A. 26)}

Sind $x,y \geq 0$ und $xy=yx$, dann ist $xy \geq 0$ (Spektralsatz, betrachte unitale, kommutative C*-Algebra $C^*(x,y,1)$). Allgemein gilt auch $\sigma(xy)\subset \sigma(xy)\cup\{0\}=\sigma(y^{1/2}xy^{1/2})\cup\{0\}\subset [0,\infty)$, aber $xy$ muss nicht s.a. sein: $\left(\begin{smallmatrix} 1 & 1 \\ 1 & 1 \end{smallmatrix}\right)\left(\begin{smallmatrix} 1 & 0 \\ 0 & 0 \end{smallmatrix}\right)=\left(\begin{smallmatrix} 1 & 0 \\ 1 & 0 \end{smallmatrix}\right)$. 

\section{Spektraltheorie in $\B(\Hc)$}

\subsection{Multiplikationsoperatoren und Borel-FK für diagonalisierbare Operatoren}

Sei $T\in \B(\Hc)$ mit ONB $(e_i)$ aus Eigenvektoren. Dann: $U: \Hc \ni e_n \mapsto \delta_n \in l^2(I)$ unitär, $M_f: l^2(I) \ni g \mapsto fg \in l^2(I)$ Multiplikationsoperator mit $f\in l^\infty(I)$ gegeben via $f(i)=\lambda_i\coloneqq T(e_i)$ und $T=U^*M_f U$ ist unitär äquivalent zu Multiplikationsoperator.

Aber: Es gibt Multiplikationsoperatoren auf $L^2$ ohne Eigenwerte: Betrachte $M_x \in L^2([0,1])$, dann für $\lambda\in\C$ wegen $(x-\lambda)=0$ fast überall: $x f(x) = \lambda f(x)$ impliziert $f=0$ fast überall.

Ein Operator $T\in \B(\Hc)$ heißt \emph{diagonalisierbar}, falls es lokalisierbaren Maßraum $(\Omega, \Sigma, \mu)$ sowie $f\in L^\infty(\Omega, \mu)$ gibt mit $T$ unitär äquivalent zum Multiplikationsoperator $M_f$ auf $L^2(\Omega, \mu)$.

\underline{Ziel}: Zeige, dass jeder normale Operator $T\in \B(\Hc)$ diagonalisierbar ist, entwickle beschränkten Funktionalkalkül für Multiplikationsoperatoren und lifte diesen auf normale Operatoren via Multiplikatordarstellung hoch.

\subsubsection{$C^*$ Algebra der Multiplikationsoperatoren}

Die Abbildung $L^\infty \ni f \mapsto M_f \in \B(L^2)$ ist isometrischer, einserhaltender *-Homomorphismus, also Isomorphismus auf sein Bild $\M\coloneqq \M(L^2(\Omega,\mu))=\{ M_f: f\in L^\infty \}$. Somit ist $\M$ kommutative \emph{$C^*$-Algebra der Multiplikationsoperatoren}.

\subsubsection{Borel-Funktionalkalkül für Multiplikationsoperatoren}

Sei $f\in L^\infty$, $K\coloneqq \essrange f = \sigma(M_f)$ kompakt, dann ist $B_b(K)$ die $C^*$-Algebra der beschränkten Borelfunktionen auf $K$. 

Die Zuordnung $B_b(K) \ni g \mapsto g\circ f \in L^\infty$ ist *-Homomorphismus, aber im Allgemeinen weder injektiv (wähle auf $[0,1]$ zu $f=\id$ ein $g=0$ f.ü., aber $g\neq 0$), noch surjektiv (wähle auf $[0,2]$ mit Lebesgue-Maß $f\equiv 1$ und $g=\chi_{[0,1]}$).

Definiere \emph{Borel-Funktionalkalkül für $M_f$} via einserhaltendem *-Homomorphismus $B_b(\sigma(M_f)) \to L^\infty \to \B(L^2)$ und schreibe $g(M_f)$ für $M_f$ eingesetzt in $g$. 

\subsubsection{Borel-Funktionalkalkül für diagonalisierbare Operatoren}

Es sei $T\in \B(\Hc)$ diagonalisierbar mit $T=U^*M_f U$, dann definiert der *-Homomorphismus $B_b(K)\to L^\infty \to \M(L^2) \subset \B(L^2) \to \B(\Hc)$ mit $g\mapsto g\circ f \mapsto M_{g\circ f} \mapsto U^* M_{g\circ f} U$ den Borel-FK für $T$ und dieser setzt den stetigen FK von $T$ fort ($\id(T)=T$, also stimmen stetiger FK und Borel-FK auf Polynomen überein und jene sind dicht in $C(K)$; beachte dass $\|\cdot\|_\infty$ Norm auf $B_b(K)$).

\subsection{normale Operatoren sind diagonalisierbar}

\subsubsection{zyklische Vektoren und invariante Teilräume}

Sei $x\in \Hc$, $T\in \B(\Hc)$, $\Kc \subset \Hc$ abgeschlossen, $\A \subset \B(\Hc)$ Algebra und $\Sc \subset \B(\Hc)$ abgeschlossen unter Adjunktion.

Dann heißt $x$ \emph{zyklischer Vektor für $\A$}, falls $\A x$ dicht in $\Hc$. Ferner heißt $x$ \emph{zyklischer Vektor für $T$}, falls $x$ zyklisch für $C^*(T,1)$. Es heißt $K$ \emph{invarianter Teilraum von $T$}, falls $T\Kc \subset \Kc$ und \emph{invarianter Teilraum von $\Sc$}, falls $\Kc$ invariant für alle $S\in \Sc$. 

Ist $\Kc$ invariant unter $\Sc$, so auch $\Kc^\bot$ (wegen Abgeschlossenheit unter Adjunktion!) und die orthogonale Projektion $P_\Kc$ auf $\Kc$ kommutiert mit allen Elementen aus $\Sc$ (vgl. VNA Kommutante: $\Kc$ invarianter Teilraum gdw. $P_\Kc\in \Sc'$).

\subsubsection{zyklische Vektoren von Matrizen (Ü.A. 40)}

Sei $T$ eine s.a. Matrix. Dann besitzt $T$ genau dann eine zyklischen Vektor, wenn die Eigenwerte von $0$ verschieden und einfach sind. Man zeigt $\Rc(T)\neq \C^n$ und wegen $\Rc(T^k) \subset \Rc(T)$ folgt dann, dass $\{T^k\eta: k\geq 0\}$ für kein $\eta$ total sein kann. Nutze dazu $\Rc(T)=\Nc(T)^\bot\neq \C^n$ im Fall $\lambda = 0$ und $\dim(\Rc(TP_\lambda^\bot))\leq \dim(E_\lambda) \leq n-2$ für einen mehrfachen Eigenwert $\lambda$. Umgekehrt ist $\xi=\eta_1+\dots+\eta_n$ für $(\eta_i)$ ONB aus nicht-entarteten Eigenvektoren zu Eigenwerten ungleich Null ein zyklischer Vektor (zeige induktiv $\eta_i$ und $\eta_{i+1}+\dots+\eta_n$ liegen im Abschluss der erzeugten Algebra).

\subsubsection{Spektralmaß $\mu_x$}

Für $f\in C(K)$ mit $f \geq 0$ ist $f(T) \geq 0$, somit $\langle f(T)x, x \rangle \geq 0$, also $C(K)\ni f \mapsto \langle f(T)x,x \rangle \in \C$ positives, stetiges Funktional auf $C(K)$.

Nach Riesz-Markov existiert ein eindeutiges reguläres Borelmaß $\mu_x$ mit $\int_K f \,\mathrm{d}\mu_x = \langle f(T)x, x \rangle$. Das Maß $\mu_x$ heißt das zu $x$ gehörige \emph{Spektralmaß}.

% TODO (Warum) Stimmt die Identität für B_b(K)?

\subsubsection{Beispiel eines Spektralmaßes (Ü.A. 36)}

Betrachte zu $R=\left(\begin{smallmatrix} 0 & 1 \\ 1 & 0 \end{smallmatrix}\right)$ den zyklischen Vektor $(1,0)$. Es gilt $K\coloneqq \sigma(R)=\{-1, 1\}$. Wegen $\|(1,0)\|^2=1$ gilt $\mu(K)=1$ und wegen $$0=\langle \left(\begin{smallmatrix} 0 \\ 1 \end{smallmatrix}\right), \left(\begin{smallmatrix} 1 \\ 0 \end{smallmatrix}\right) \rangle=\langle \left(\begin{smallmatrix} 0 & 1 \\ 1 & 0 \end{smallmatrix}\right)\left(\begin{smallmatrix} 1 \\ 0 \end{smallmatrix}\right), \left(\begin{smallmatrix} 1 \\ 0 \end{smallmatrix}\right) \rangle=\int_K x \, \mathrm{dx}=\mu(\{1\})-\mu(\{-1\})$$ folgt $\mu(\{1\})=\mu(\{-1\})=1/2$.


\subsubsection{Multiplikatordarstellung für normale Operatoren}

Sei $T\in \B(\Hc)$ normal, $K\coloneqq \sigma(T)$, $x\in \Hc$ zyklisch für $T$. 

Es gilt für $f,g\in C(K)$: $\langle f(T)x, g(T)x \rangle = \langle f,g\rangle_{L^2}$, also $\tilde U: \Hc\ni f(T)x \mapsto f \in C(K)$ wohldefiniert und isometrisch. Da $x$ zyklisch für $T$, d.h. $\{ f(T)x: f\in C(K) \}$ dicht in $\Hc$, und $C(K)$ dicht in $L^2$, besitzt $\tilde U$ Fortsetzung zu unitärem Operator $U: \Hc \to L^2$ mit $U(x)=U(1(T)x)=1$ und $U^*TU=M_\mathrm{id}$.

Gibt es keinen zyklischen Vektor für $T$, so zerlege $\Hc$ in direkte Summe orthogonaler, zyklischer Teilräume (Zorn!) und wende obigen Fall auf die Räume der Zerlegung an.

\subsubsection{Beispiel einer konkreten Multiplikatordarstellung (Ü.A. 3)}

Betrachte den zweiseitigen Rechtsshift $S: l^2(\Z) \to l^2(\Z)$. Dieser besitzt keine Eigenvektoren (sonst leite Rekursionsformel für die Folge her und erhalte Widerspruch zu $\lim_{|n|\to \infty} x_n = 0$). Für $\lambda \in \mathbb{T} \subset \C$ konstruiere approximativen Eigenwert von $S$ durch abschneiden der $l^\infty(\Z)$-Folge $(\lambda^{-n})_{n\in \Z}$, also $\sigma(S) \supset \mathbb{T}$. Erhalte via diskreter Fouriertransformation $\F: l^2(\Z) \to L^2(\mathbb T)$ den Operator $\F S\F^*$ auf $L^2(\mathbb T)$. Es gilt $\F S\F^*=M_{e^{i\varphi}}$. Daher $\sigma(S)=\essrange(e^{i\varphi})=\mathbb{T}$.

\subsubsection{Borel-FK für normale Operatoren}

Da normale Operatoren diagonalisierbar sind, kann der Borel-Funktionalkalkül für diagonalisierbare Operatoren verwendet werden.

\subsection{Zerlegung des Spektrums}

Für $T\in \B(\Hc)$ lässt sich das Spektrum $\sigma(T)$ disjunkt zerlegen als $\sigma(T)=\sigma_p(T) \cup \sigma_c(T) \cup \sigma_r(T)$. Es ist $\lambda \in \sigma_p(T)$, falls $T-\lambda$ nicht injektiv. Es ist $\lambda \in \sigma_c(T)$, falls $T-\lambda$ injektiv, nicht surjektiv, aber $\Rc(T-\lambda)$ dicht und $\lambda \in \sigma_r(T)$, falls $T-\lambda$ injektiv, nicht surjektiv und $\Rc(T-\lambda)$ nicht dicht.

\subsubsection{Projektionswertiges Maß und Spektralscharen}

Sei $T\in \B(\Hc)$ normal, dann definiert $\mathbb{P}: \B(\C) \ni A \mapsto \chi_{A\cap \sigma(T)}(T) \in P(\Hc)$ ein projektionswertiges Maß ($\mathbb{P}(\emptyset) = 0$, $\mathbb{P}(\C) = 1$, $\sigma$-Additivität gilt stop). Im Fall $T=T^*$ ist $\sigma(T)\subset \R$ und es reicht $\lambda \mapsto P_\lambda \coloneqq \mathbb{P}((-\infty, \lambda])$ zu betrachten, genannt \emph{Spektralschar}. Die Zuordnung $\lambda \mapsto P_\lambda$ ist monoton und rechtsseitig stop-stetig. Wir benutzen sie, um Aussagen über die Zusammensetzung des Spektrums zu zeigen.

\subsubsection{Das Spektrum normaler und s.a. Operatoren}

Ist $T\in \B(\Hc)$ normal, dann $\sigma_r(T)=\emptyset$, denn (O.B.d.A $\lambda=0$, sonst betrachte $\tilde A \coloneqq A-\lambda$ normal) ist $\Rc(T)$ nicht dicht, dann wegen $\Nc(T)=\Nc(T^*)$ (denn $\|Tx\|=\|T^*x\|$ für $T$ normal) folgt $\{0\} \neq \Rc(T)^\bot = \Nc(T^*) = \Nc(T)$, also $\lambda=0$ Eigenwert.

Nun: $T=T^*$, dann $\lambda \mapsto P_\lambda$ auf $\sigma(T)$ strikt monoton wachsend, in $\lambda \in \sigma_c(T)$ beidseitig stop-stetig (deshalb kontinuierlich!), d.h. zusätzlich links-stop-stetig,  und mit Sprungstellen in $\lambda \in \sigma_p(T)$. Bild!

Ein $\lambda\in \C$ heißt \emph{approximativer Eigenwert}, falls es Folge $(x_n)$ von Einheitsvektoren gibt mit $\|(T-\lambda)x_n\| \to 0$, insbesondere gilt für die Fourierkoeffizienten: $\langle Tx_n, x_n \rangle \to \lambda$. Das Spektrum eines normalen Operators besteht vollständig aus approximativen Eigenwerten.

\section{unbeschränkte Operatoren}

\subsection{Begriffsbildung}

Ein unbeschränkter Operator auf $\Hc$ ist ein Operator $T:\Hc \supset D(T) \to \Hc$, der nicht notwendigerweise beschränkt sein muss. Beispiele: 1) Betrachte auf $\Hc\coloneqq L^2([1,\infty))$ den Operator $T_1\coloneqq M_x$, $g\coloneqq x^{-1}$, dann $g\in \Hc$, aber $T_1g\not\in \Hc$, also $g\not\in D(T_1)$. Definiert man $g_n$ wie $g$ mit $n$ statt $\infty$, so zeigt die Folge die Unbeschränktheit.  2) Nun betrachte auf $L^2([0,1])$ den Operator $T_2\coloneqq -i \frac{d}{dx}$, dieser ist erstmal nur sinnvoll definiert für differenzierbare Funktionen und $e_n\coloneqq e^{inx}$ zeigt Unbeschränktheit: $\|e_n\|=1$, aber $\|T_2 e_n\|=n$. Mit $D(T_2)\coloneqq \{f\in C^1([0,1]): f(0)=f(1) \}$ ist $T_2$ (ebenso wie $T_1$) symmetrisch ($\langle Tx,y\rangle = \langle x, Ty \rangle$ für $x,y\in D(T)$). Verbindung: Hellinger-Töplitz!

Es bezeichne $\G(T)$ den Graphen von $T$. Statte $\G(T)$ mit $1$-oder $2$-direkter Summennorm aus. Auf $D(T)$ definiere die \emph{Graphennorm} $\|x\|_T \coloneqq \|(x,Tx)\|_{\G(T)}$. Dann ist $T: (D(T), \|\cdot\|_T) \to \Hc$ stetig.

Definiere Ordnung auf den unbeschränkten Operatoren via $T_1 \subset T_2$, falls $D(T_1)\subset D(T_2)$ und $T_2|_{D(T_1)} = T_1$. Daraus ergeben sich die Begriffe \emph{Fortsetzung} und \emph{Einschränkung}.

Für zwei Operatoren $S,T$ definiere die Summe $S+T$ mit $D(S+T)=D(S)\cap D(T)$ und die Komposition $ST$ mit $D(ST)=\{x\in D(T): Tx\in D(S)\}$. Die unbeschränkten Operatoren bilden keine Algebra (nicht mal einen V.R.) aufgrund der Problematik der Definitionsbereiche (Nullvektor ist $0$ auf $\Hc$, ist $T$ Operator mit $D(T)\subsetneqq \Hc$, dann gilt für alle Operatoren $S$ aber $D(T+S)\subsetneqq \Hc$, also kann kein $S$ invers zu $T$ sein).

Ein Operator $T$ heißt \emph{abgeschlossen}, falls sein Graph abgeschlossen ist, also $x_n\to x$ und $Tx_n\to y$ impliziert $x\in D(T)$ und $Tx=y$. Äquivalent dazu ist: $(\G(T),\|\cdot\|_{\G(T)})$ bzw. $(D(T), \|\cdot\|_T)$ ist B.R.

Eine Teilmenge $\G \subset \Hc \oplus \Kc$ ist genau dann Graph eines Operators, falls aus $(0,y)\in \G$ stets $y=0$ folgt ($Tx=y$ für $(x,y)\in \G$ definiert auf $\{x\in \Hc: \text{ es ex } y\in \Kc \text{ mit } (x,y)\in \G\}$ einen linearen Operator). Es sind äquivalent, dass $\overline{\G(T)}$ Graph eines Operators $\overline{T}$ ist, dass $T$ eine abgeschlossene Fortsetzung besitzt und dass $\overline{\G(T)}$ kein Element der Form $(0,y)$ mit $y\neq 0$ enthält. Ein solcher Operator heißt \emph{abschließbar} und die kleinste abgeschlossene Fortsetzung $\overline{T}$ heißt \emph{Abschluss} von $T$. Beispiele: 1) $f\in L^2$, betrachte Multiplikationsoperator $M_f$ auf maximalem Definitionsbereich $D$. Dann ist $M_f$ abgeschlossen. 2) Der Ableitungsoperator $P\coloneqq -i \frac{d}{dx}: L^2(\R) \supset C_c^\infty(\R)$ ist abschließbar, denn $\F P\F^{-1} = M_x$ ist abschließbar. Man kann $H^1(\R)$ daher auch als Definitionsbereich von $\overline{P}$ definieren. 3) $T_\delta: L^2(\R) \supset C_c(\R) \ni f \mapsto f(0) \in \C$ ist nicht abschließbar: Nutze Hutfunktionen mit $f(1)=1$, aber $f\to 0$ in $L^2$.

\subsection{Adjungierte}

Zu $T:\Hc \supset D \to \Kc$ d.d. wollen wir die Adjungierte definieren. Es soll $\langle Tx,y \rangle = \langle x, T^*y \rangle$ für $x\in D$ und $y\in D^*\coloneqq D(T^*)$ gelten. Sofern $x\mapsto \langle Tx, y \rangle$ stetige LF (auf $\Hc$ wegen d.d.) ist (da $T$ unbeschränkt, ist dies nicht für alle $y$ der Fall), liefert R.F. $T^*x\coloneqq z$ wobei $\langle Tx,y \rangle = \langle x, z \rangle$ gilt. Man nennt $T^*$ die \emph{Adjungierte} von $T$. Ist $T: \Hc \supset D \to \Hc$ und $T \subset T^*$, dann heißt $T$ \emph{symmetrisch} und gilt $T^*=T$, so heißt $T$ \emph{selbstadjungiert}. 

Beispiel: Sei $f\in L^2$, dann $M_f: L^2\supset D \to L^2$ Multiplikationsoperator. Es ist $M_{\overline{f}}: L^2 \supset D \to L^2$ die Adjungierte von $M_f$. Ist $f$ reell, so ist $M_f$ sicher symmetrisch und ist $D$ der maximale Definitionsbereich von $M_f$, dann auch s.a.

Folgender blauer Werkzeugkasten (Hauptsatz) steht zur Verfügung: Ist $T_0 \subset T$, dann $T^* \subset T_0^*$, denn für $y\in D(T^*)$ ist $x\mapsto \langle T_0 x, y \rangle = \langle Tx, y\rangle = \langle x, T^* y \rangle$ stetige LF. Für den unitären Operator $U: \Hc \oplus \Kc \ni x \oplus y \mapsto y \oplus -x \in \Kc \oplus \Hc$ gilt: $\G(T^*) = (U\G(T))^\perp$, somit ist $T^*$ insbesondere abgeschlossen. Es ist $T$ genau dann abschließbar, wenn $T^*$ d.d. (nutze $0\oplus z \in \overline{\G(T)}$ gdw. $z\in D(T^*)^\perp$ und Kriterium für Abschließbarkeit). Ist $T$ abschließbar, so gilt $\overline{T} = T^{**}$ (nachrechnen) und $T^* = \overline{T}^*$, denn $T^*=\overline{T^*}=T^{***}=\overline{T}^*$.

Beispiel (Reed, Simon, VIII Beispiel 8): Sei $f$ beschränkt, messbar, aber $f\not\in L^2(\R)$ und $g\in L^2(\R)$. Definiere $T\varphi \coloneqq \langle \varphi, f \rangle g$ auf $D(T)\coloneqq \{ \varphi\in L^2(\R): f\varphi \in L^1(\R) \}$. Da $f$ beschränkt, folgt $C_c(\R) \subset D(T)$, also $T$ d.d. Für $\psi \in D(T^*)$ gilt dann $T^*\psi = \langle \psi, g \rangle f$ und da $f\not \in L^2$, folgt $D(T^*) \subset \{ g \}^\perp$, somit $T^*$ nicht d.d. und somit $T$ nicht abschließbar. Leichteres Beispiel für nicht abschließbaren Operator, aber zwischen verschiedenen Hilberträumen: Punktauswertung, nutze Hutfunktionen, die in $L^2$ gegen $0$ gehen.

\subsubsection{Hauptkriterium}

Ist $T$ d.d., abg. und symmetrisch, so sind äquivalent: 1) $T$ s.a. 2) $\Nc(T^*\pm i)=\{0\}$ 3) $\Rc(T\pm i)$ dicht in $\Hc$ 4) $\Rc(T\pm i)=\Hc$. In 1) bis 3) sind immer beide Vorzeichen gefordert (da man $\Nc(T\pm i)=\Rc(T^*\mp i)^\perp$ benutzen möchte). Man nutzt im Beweis, dass wegen $T$ symmetrisch die Identität $\|(T\pm i)x\|^2=\|x\|_T$ gilt. Kriterium 3) ist gut zum Nachweis geeignet, da man hierzu nicht die Adjungierte kennen muss.

\subsubsection{Beispiele zur Nutzung des Hauptkriteriums}

% TODO hinzufügen

\subsection{Spektraltheorie unbeschränkter Operatoren}

Wir wollen den Spektralsatz und Funktionalkalkül von $\B(\Hc)$ hochziehen. Wir beschränken uns auf den s.a. Fall, obwohl das Resultat auch für normale Operatoren gilt. Um unsere bisherigen Resultate zu verwenden, werden wir einen unbeschränkten s.a. Operator via Cayley-Transformation in einen beschränkten Operator überführen.

Für nicht abgeschlossene Operatoren ist $\sigma(T)=\emptyset$, also betrachte nur abgeschlossene Operatoren. Dann sind Spektrum, Resolvente usw. wie gewohnt definiert. Ein d.d. und abg. Operator $T$ ist genau dann beschränkt invertierbar, wenn es $T^*$ ist, insbesondere gilt $\sigma(T^*)=\overline{\sigma(T)}$ (Beweis: nachrechnen, zeige $(T^*)^{-1}=(T^{-1})^*$ via $\langle (T^{-1})^*T^*x,y \rangle = \langle x, y \rangle$ für $x\in D(T^*), y\in \Hc$ und umgekehrte Reihenfolge).

Ist $T$ d.d., abg., symmetrisch, so sind s.a. und $\sigma(T)\subset \R$ äquivalent: Nutze das Hauptkriterium! Ist $T$ nicht s.a., so ist $i$ oder $-i$ Eigenwert, also $\sigma(T) \not\subset \R$. Ist $T$ s.a. und $\lambda=\alpha + i\beta$ mit $\beta\neq 0$, so gilt $T-\lambda = \beta(\frac{T-\alpha}{\beta}-i)$ (hier braucht man $\beta\neq 0$!), $\frac{T-\alpha}{\beta}$ s.a., also $T-\lambda$ bijektiv (und damit beschränkt invertierbar!), also $\lambda \not\in \sigma(T)$.

\subsubsection{Cayley-Transformation}

Definiere die Cayley-Abbildung $c: \C\setminus \{-i\} \ni z \mapsto \frac{z-i}{z+i} \in \C\setminus \{1\}$. Diese ist bijektiv und bildet $\R$ auf $\mathbb{T} \setminus \{1\}$ ab. Weitere Werte: $z(0)=-1, z(1)=-i, z(-1)=i$ sowie $\lim_{x\to \pm \infty} c(x) = 1$ (Skizze für die Wirkung auf der oberen Halbebene siehe Wikipedia). Für s.a. $T$ definiere $U_T\coloneqq (T-i)(T+i)^{-1}$ (vgl. Cayley-Abbildung!). Es ist $U_T$ unitär (nutze $\|(T+i)x\|=\|(T-i)x\|$, vgl. Beweis Hauptkriterium für Selbstadjungiertheit). Aus $T_1\neq T_2$ folgt $U_{T_1}\neq U_{T_2}$ (vgl. Ü.A., nutze, dass $T$ aus $U_T$ rekonstruierbar ist, nutze $1-U_T$ bijektiv). Es gilt $\lambda\in \sigma(T)$ gdw. $c(\lambda)\in \sigma(U_T)$. Ist $\sigma(T)=\R$, so folgt (weil $\sigma(U_T)$ abgeschlossen), dass $1\in \sigma(U_T)$, aber $1$ ist kein Eigenwert von $U_T$ (wichtig für den Spektralsatz!).

\subsubsection{Spektralsatz und Funktionalkalkül}

Es ist äquivalent, dass $T$ s.a. ist und dass $T$ unitär äquivalent zu einem Multiplikationsoperator $M_f$ (aber hier nicht notwendigerweise $f\in L^\infty$!) mit $f$ reell ist. Rückrichtung ist klar. Aus dem Spektralsatz für normale, beschränkte Operatoren erhalte $V^*U_TV=M_u$ (denn $U_T$ ist beschränkt und wegen unitär auch normal) und es gilt $\Rc(u)\subset \mathbb T$ wegen $u$ unitär. Da $1\not\in \sigma_p(U_T)$, gilt $\mu(\{\omega\in \Omega: u(\omega)=1\})=0$ (sonst entsprechende charakteristische Funktion Eigenvektor), also $u(\omega)\in D(c^{-1})$ fast überall. Setze $f(\omega)\coloneqq c^{-1}(u(\omega))$, dann $\Rc(f) \subset \R$, also $f$ reell und $M_u$ ist Cayley-Transformation von $M_f$ (nachrechnen, nutze $M_f-i=M_{f-i}$ usw.), also $U_T$ Cayley-Trafo von $V^*M_fV$ (nachrechnen, schreibe $i=V^*iV$), aber auch von $T$, also wegen Eindeutigkeit $T=V^*M_fV$.

Für $T=V^*M_fV$ und $g:\sigma(T)\to \C$ messbar (nicht notwendigerweise beschränkt!) definiere Funktionalkalkül via $g(T)=V^*M_{g\circ f}V$.

\end{document}
