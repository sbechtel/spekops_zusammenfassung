\documentclass[11pt,a4paper]{scrartcl}

\usepackage[utf8]{inputenc}
\usepackage[T1]{fontenc}
\usepackage[ngerman]{babel}
\usepackage{amsmath,amsthm,amssymb,dsfont}
\usepackage{mathtools}
\usepackage[paper=a4paper,left=25mm,right=25mm,top=25mm,bottom=25mm]{geometry}
\usepackage{float}
\usepackage{hyperref}
\usepackage{enumerate}
\usepackage{url}
\usepackage{tikz}
\usepackage{esint}
\usepackage{csquotes}
\usepackage{textcomp}

\usepackage{setspace}

\parindent 0pt
\linespread{1.5}

% Makros

\newcommand{\N}{\mathbb{N}} % natuerliche Zahlen
\newcommand{\Z}{\mathbb{Z}} % ganze Zahlen
\newcommand{\Q}{\mathbb{Q}} % rationale Zahlen
\newcommand{\R}{\mathbb{R}} % reelle Zahlen
\newcommand{\K}{\mathbb{K}} % Körper
\newcommand{\C}{\mathbb{C}} % komplexe Zahlen
\newcommand{\D}{\mathcal{D}}
\newcommand{\E}{\mathcal{E}}
\newcommand{\Hc}{\mathcal{H}}
\newcommand{\Sc}{\mathcal{S}}
\newcommand{\Kc}{\mathcal{K}}
\newcommand{\A}{\mathcal{A}}
\newcommand{\B}{\mathcal{B}}
\newcommand{\M}{\mathcal{M}}
\newcommand{\Nc}{\mathcal{N}}
\newcommand{\F}{\mathcal{F}}
\newcommand{\norm}[1]{\|#1\|}
\newcommand{\laplace}{\triangle}
\newcommand{\circum}{\text{\textasciicircum}}

% Umgebungen für Definitionen, Sätze, usw.

\theoremstyle{plain}
\newtheorem{thm}{Satz}[section]
\newtheorem*{lem}{Lemma}
\newtheorem{cor}[thm]{Korollar}
\newtheorem{prop}[thm]{Proposition}
\newtheorem*{ex}{Beispiel}
\newtheorem*{ntion}{Notation}

\theoremstyle{definition}
\newtheorem{defn}[thm]{Definition}

\theoremstyle{remark}
\newtheorem*{bem}{Bemerkung}
\newtheorem{bemnumber}[thm]{Bemerkung}

\def\Satzrefname{Satz}

\DeclareMathOperator{\supp}{supp}
\DeclareMathOperator{\esssupp}{ess supp}
\DeclareMathOperator{\essrange}{ess range}
\DeclareMathOperator{\id}{id}
\DeclareMathOperator{\loc}{loc}
\DeclareMathOperator{\pv}{pv}
\DeclareMathOperator{\grad}{grad}

\begin{document}

\title{Zusammenfassung Spektraltheorie und Operatoralgebren}
\author{Sebastian Bechtel}
\maketitle

\section{Spektraltheorie in $\B(\Hc)$}

\subsection{Multiplikationsoperatoren und Borel-FK für diagonalisierbare Operatoren}

Sei $T\in \B(\Hc)$ mit ONB $(e_i)$ aus Eigenvektoren. Dann: $U: \Hc \ni e_n \mapsto \delta_n \in l^2(I)$ unitär, $M_f: l^2(I) \ni g \mapsto fg \in l^2(I)$ Multiplikationsoperator mit $f\in l^\infty(I)$ gegeben via $f(i)=\lambda_i\coloneqq T(e_i)$ und $T=U^*M_f U$ ist unitär äquivalent zu Multiplikationsoperator.

Aber: Es gibt Multiplikationsoperatoren auf $L^2$ ohne Eigenwerte: Betrachte $M_x \in L^2([0,1])$, dann für $\lambda\in\C$ wegen $(x-\lambda)=0$ fast überall: $x f(x) = \lambda f(x)$ impliziert $f=0$ fast überall.

Ein Operator $T\in \B(\Hc)$ heißt \emph{diagonalisierbar}, falls es lokalisierbaren Maßraum $(\Omega, \Sigma, \mu)$ sowie $f\in L^\infty(\Omega, \mu)$ gibt mit $T$ unitär äquivalent zum Multiplikationsoperator $M_f$ auf $L^2(\Omega, \mu)$.

\underline{Ziel}: Zeige, dass jeder normale Operator $T\in \B(\Hc)$ diagonalisierbar ist, entwickle beschränkten Funktionalkalkül für Multiplikationsoperatoren und lifte diesen auf normale Operatoren via Multiplikatordarstellung hoch.

\subsubsection{$C^*$ Algebra der Multiplikationsoperatoren}

Die Abbildung $L^\infty \ni f \mapsto M_f \in \B(L^2)$ ist isometrischer, einserhaltender *-Homomorphismus, also Isomorphismus auf sein Bild $\M\coloneqq \M(L^2(\Omega,\mu))=\{ M_f: f\in L^\infty \}$. Somit ist $\M$ kommutative \emph{$C^*$-Algebra der Multiplikationsoperatoren}.

\subsubsection{Borel-Funktionalkalkül für Multiplikationsoperatoren}

Sei $f\in L^\infty$, $K\coloneqq \essrange f = \sigma(M_f)$ kompakt, dann ist $B_b(K)$ die $C^*$-Algebra der beschränkten Borelfunktionen auf $K$. 

Die Zuordnung $B_b(K) \ni g \mapsto g\circ f \in L^\infty$ ist *-Homomorphismus, aber im Allgemeinen weder injektiv (wähle $g=0$ f.ü., aber $g\neq 0$), noch surjektiv (wähle auf $[0,2]$ mit Lebesgue-Maß $f\equiv 1$ und $g=\chi_{[0,1]}$).

Definiere \emph{Borel-Funktionalkalkül für $M_f$} via einserhaltendem *-Homomorphismus $B_b(\sigma(M_f)) \to L^\infty \to \B(L^2)$ und schreibe $g(M_f)$ für $M_f$ eingesetzt in $g$. 

\subsubsection{Borel-Funktionalkalkül für diagonalisierbare Operatoren}

Es sei $T\in \B(\Hc)$ diagonalisierbar mit $T=U^*M_f U$, dann definiert der *-Homomorphismus $B_b(K)\to L^\infty \to \M(L^2) \subset \B(L^2) \to \B(\Hc)$ mit $g\mapsto g\circ f \mapsto M_{g\circ f} \mapsto U^* M_{g\circ f} U$ den Borel-FK für $T$ und dieser setzt den stetigen FK von $T$ fort ($id(T)=T$, also stimmen stetiger FK und Borel-FK auf Polynomen überein und jene sind dicht in $C(K)$; beachte dass $\|\cdot\|_\infty$ Norm auf $B_b(K)$).

\subsection{normale Operatoren sind diagonalisierbar}

\subsubsection{zyklische Vektoren und invariante Teilräume}

Sei $x\in \Hc$, $T\in \B(\Hc)$, $\Kc \subset \Hc$ abgeschlossen, $\A \subset \B(\Hc)$ Algebra und $\Sc \subset \B(\Hc)$ abgeschlossen unter Adjunktion.

Dann heißt $x$ \emph{zyklischer Vektor für $\A$}, falls $\A x$ dicht in $\Hc$. Ferner heißt $x$ \emph{zyklischer Vektor für $T$}, falls $x$ zyklisch für $C^*(T,1)$. Es heißt $K$ \emph{invarianter Teilraum von $T$}, falls $T\Kc \subset \Kc$ und \emph{invarianter Teilraum von $\Sc$}, falls $\Kc$ invariant für alle $S\in \Sc$. 

Ist $\Kc$ invariant unter $\Sc$, so auch $\Kc^\bot$ (wegen Abgeschlossenheit unter Adjunktion!) und die orthogonale Projektion $P_\Kc$ auf $\Kc$ kommutiert mit allen Elementen aus $\Sc$ (vgl. VNA: Kommutante).

\subsubsection{Spektralmaß $\mu_x$}

Für $f\in C(K)$ mit $f \geq 0$ ist $f(T) \geq 0$, somit $\langle f(T)x, x \rangle \geq 0$, also $C(K)\ni f \mapsto \langle f(T)x,x \rangle \in \C$ positives, stetiges Funktional auf $C(K)$.

Nach Riesz-Markov existiert ein eindeutiges reguläres Borelmaß $\mu_x$ mit $\int_K f \,\mathrm{d}\mu_x = \langle f(T)x, x \rangle$. Das Maß $\mu_x$ heißt das zu $x$ gehörige \emph{Spektralmaß}.

% TODO (Warum) Stimmt die Identität für B_b(K)?

\subsubsection{Multiplikatordarstellung für normale Operatoren}

Sei $T\in \B(\Hc)$ normal, $K\coloneqq \sigma(T)$. 

Ist $x\in \Hc$ zyklischer Vektor für $T$, dann gilt für $f,g\in C(K)$: $\langle f(T)x, g(T)x \rangle = \langle f,g\rangle_{L^2}$, also $\tilde U: \Hc\ni f(T)x \mapsto f \in C(K)$ wohldefiniert und isometrisch. Da $x$ zyklisch für $T$, d.h. $\{ f(T)x: f\in C(K) \}$ dicht in $\Hc$, und $C(K)$ dicht in $L^2$, besitzt $\tilde U$ Fortsetzung zu unitärem Operator $U: \Hc \to L^2$ mit $U(x)=U(\mathrm{id}(T)x)=\mathrm{id}$ und $U^*TU=M_\mathrm{id}$.

Gibt es keinen zyklischen Vektor für $T$, so zerlege $\Hc$ in direkte Summe orthogonaler, zyklischer Teilräume (Zorn!) und wende obigen Fall auf die Räume der Zerlegung an.

\subsubsection{Borel-FK für normale Operatoren}

Da normale Operatoren diagonalisierbar sind, kann der Borel-Funktionalkalkül für diagonalisierbare Operatoren verwendet werden.

\end{document}
